% Created 2011-01-05 Ср. 20:59
\documentclass[12pt, russian, oneside, article]{ncc}
\usepackage[utf8]{inputenc}
\usepackage[T1]{fontenc}
\usepackage{fixltx2e}
\usepackage{graphicx}
\usepackage{longtable}
\usepackage{float}
\usepackage{wrapfig}
\usepackage{soul}
\usepackage{textcomp}
\usepackage{marvosym}
\usepackage{wasysym}
\usepackage{latexsym}
\usepackage{amssymb}
\usepackage{hyperref}
\tolerance=1000
\usepackage[math]{pscyr}
\usepackage{indentfirst}
\providecommand{\alert}[1]{\textbf{#1}}
\begin{document}



\title{Информационная безопасность систем и технологий}
\author{Максим Захаров}
\date{05 Январь 2011}
\maketitle

\setcounter{tocdepth}{3}
\tableofcontents
\vspace*{1cm}

\href{file://~/Documents/Git/lectures/other~/Documents/Git/zamal.github.com/pdf/IBST_Lectures.pdf}{Скачать в PDF}

\section{Протокол IP}
\label{sec-1}

Протокол сетевого уровня, обеспечивает передачу дейтаграмм между компьютерами, разбиение их на фрагменты и её маршрутизацию.

Протокол IP не гарантирует надёжную доставку дейтаграмм, он не управляет потоком данных и не выявляет ошибки передачи. Оптимизация маршрута проводится только для соседних узлов.

Максимальный размер пакета равен 2$^{\mathrm{16}}$ байт.

Связь по протоколу поддерживается без установления соединения.

Протокол IP использует 32-разрядные адреса компьютера в версии 4. В версии 6 используются 128-разрядные адреса.
\subsection{Заголовок протокола IP}
\label{sec-1_1}

Длина заголовка равна 5 или 6 32-разрядных слов.

Поля:

\begin{enumerate}
\item \emph{Версия}. В нём содержится версия протокола IP. Это поле указывает программе-получателю как декодировать все остальные поля заголовка, т. к в разных версиях поля отличаются. Если программа не работает с указанной версией протокола, то дейтаграмма отбрасывается.
\item \emph{Длина заголовка}. В нём хранится длина заголовка. Размер поля 4 бита. В нём написано либо число 5, либо 6.
\item \emph{Тип обслуживания}. Длина 8 бит. Первые 3 бита обозначают приоритет дейтаграммы. Если он равен 0 --- обычная, все 1 --- сетевое управление. 1 бит --- задержка, следующий --- пропускная способность, далее --- надёжность.
\item \emph{Длина дейтаграммы}. Длина поля 16 бит.
\item \emph{Идентификатор}. Длина 16 бит. В нём содержится уникальный идентификатор, присвоенный дейтаграмме передающим узлом. Идентификатор получают от протоколов вышестоящего уровня. Он используется для правильной сборки сообщений при фрагментации.
\item \emph{Флаги}. 3 бита. 1 бит не используется, 2 бит --- флаг не фрагментировать, 3 бит --- есть ещё фрагменты.
\item \emph{Смещение фрагмента}. Номер фрагмента в сообщении. Его длина 13 бит. Смещение указывается в единицах, кратных 8 байтам.
\item \emph{Время жизни}. Длина 8 бит. Время в секундах, которое отводится на доставку дейтаграммы. Обычно 15-30 секунд. Если время жизни истекло, дейтаграмма уничтожается. Каждый узел, в который попадает дейтаграмма, уменьшает время жизни на 1 или больше.
\item \emph{Протокол}. Длина 8 бит. В нём содержится код протокола транспортного уровня, для которого предназначена дейтаграмма. Для TCP --- 6.
\item \emph{Контрольная сумма заголовка}. Длина 16 бит. Контрольная сумма вычисляется по всему заголовку и меняется в каждом узле, через которую она проходит. Она вычисляется путём сложения всех 16-разрядных слов заголовка и дополнением результата 1.
\item \emph{IP-адрес отправителя}. 32 бита.
\item \emph{IP-адрес получателя}. 32 бита.
\item \emph{Опции}. Каждой опции отводится 1 байт. Каждый байт делится на 3 части.

\begin{itemize}
\item 1 часть --- копия, занимает 1 бит (нужно ли копировать поле ``опции'' при фрагментации дейтаграмм).
\item 2 часть --- класс опции, занимает 2 бита (класс 00 --- управление сетью, класс 10 --- отладка сети).
\item 3 часть --- номер опции, 5 бит (00000 --- конец списка опций, 00011 --- совместная маршрутизация, где часть маршрута задаётся отправителем, остальные определяются шлюзами, 01001 --- маршрутизация отправителя, где все узлы задаются отправителем, 00111 --- обозначает запись маршрута, 00100 --- запись временных меток).
\end{itemize}

\item \emph{Заполнитель}. Для дополнения заголовка до целого числа 32-разрядных слов.
\end{enumerate}

После формирования дейтаграммы конструируются заголовок. Для него вычисляется контрольная сумма. Определяется узел, в которую предполагается отправить дейтаграмму. Если получатель находится в той же сети, что и отравитель, этим следующим узлом будет получатель. Если нет, то следующим узлом будет шлюз. Если используется опция ``запись маршрута'', то добавляются адреса узлов нужного маршрута. Дейтаграмма передаётся через протоколы нижележащих уровней в сеть.

Каждый шлюз в сети, через который проходит дейтаграмма проверяет контрольную сумму. Если контрольные суммы не совпадают, дейтаграмма уничтожается, а отправителю передаётся сообщение об ошибке. Если совпадают, уменьшается значение поля ``время жизни'', если это поле равно 0, то дейтаграмма уничтожается.

Определяется следующий узел маршрута, исходя из адреса получателя или опции маршрутизации. Заголовок перестраивается заново. Если при этом требуется фрагментация и соответствующий флаг не фрагментируется, дейтаграмма уничтожается и отправителю отправляется сообщение об ошибке.

При необходимости может быть записана временная метка. После того, как дейтаграмма достигла машины получателя, проверяется контрольная сумма заголовка. Получатель ждёт сборки в течение определённого времени. Если за это время сообщение не было собрано, то все его полученные фрагменты уничтожаются, отправителю отправляется сообщение об ошибке. Если всё нормально, IP заголовков уничтожаются, сообщение передаётся на более высокий уровень, если требуется, отправителю посылается ответ.
\section{Протокол ICMP (Internet control message protocol)}
\label{sec-2}

Этот протокол сообщает отправителю об ошибках в сети. Используется совместно с IP. ICMP-дейтаграмма снабжена IP-заголовком, поэтому она в сети обрабатывается также, как обычная IP-дейтаграмма. В узлах сети ICMP-дейтаграммы обрабатываются на сетевом уровне.

Сообщения об ошибках передаются машине отправителя, а внутри ICMP-сообщение. Внутри этого ICMP-сообщения находятся IP-заголовок и первые 64 бита дейтаграммы, при передаче которой возникла ошибка.

Вид заголовка хранится в ICMP-заголовке. Заголовок состоит из 3 полей:
\begin{enumerate}
\item Тип сообщения.

\begin{itemize}
\item 0 --- эхо-ответ;
\item 8 --- эхо-запрос;
\item 3 --- адресат недостижим. Это же сообщение генерируется, если шлюзу необходимо фрагментировать дейтаграмму, а в ней установлен флаг ``не фрагментировать'';
\item 4 --- снизить скорость передачи данных;
\item 5 --- переадресовать. Служебное сообщение для шлюзов при выполнении маршрутизации;
\item 11 --- время жизни дейтаграммы истекло;
\item 12 --- неправильный параметр. Возникает при обнаружении семантической или синтаксической ошибки в IP-заголовке;
\item 13 --- запрос временной метки;
\item 14 --- отклик на запрос временной метки. Они нужны для контроля прохождения дейтаграмм через узлы сети. При этом внутри ICMP-сообщения в запросе записывается вместо IP-заголовка исходная временная метка, а в отклике к этой метке добавляется метка получения запроса шлюза и метка отправки ответа шлюзу;
\item 17 --- запрос адресной маски;
\item 18 --- отклик на запрос адресной маски. Эти сообщения применяются для тестирования определённой подсети с заданной маски.
\end{itemize}

\item Код сообщения.
\item Контрольная сумма ICMP-заголовка. Вычисляется также, как контрольная сумма IP-заголовка.
\end{enumerate}
\section{Протокол IP версии 6}
\label{sec-3}


Основное отличие от версии 4 заключается в использовании 128-битных IP-адресов. Кроме этого протокол предусматривает введение метки для контроля качества обслуживания и предотвращения фрагментации в промежуточных узлах. В этом протоколе предусматривается встроенное средство для аутентификации и шифрования данных.

Заголовок имеет длину 40 байт.
\begin{enumerate}
\item \emph{Версия}. 4 бита.
\item \emph{Приоритет}. Приоритет дейтаграммы. 4 бита.
\item \emph{Метка потока}. Длина 24  бита. При помощи этого поля помечаются дейтаграммы, для которых в маршрутизаторах сети требуется специальная обработка.
\item \emph{Длина всей IP-дейтаграммы минус длина заголовка}. 16 бит.
\item \emph{Следующий заголовок}. Его длина 8 бит. В нём определяется заголовок, который находится за заголовком IP. Следующим заголовком может быть заголовок транспортного уровня либо заголовок расширения IP.
\item \emph{Предельное число транзитов}. Длина 8 бит.
\item \emph{Адрес источника}. 128 бит.
\item \emph{Адрес получателя}. 128 бит.
\end{enumerate}
\subsection{Заголовки расширений IP}
\label{sec-3_1}


\begin{enumerate}
\item Заголовок параметров транзита. В нём содержится дополнительная информация для маршрутизаторов. Используется в настоящее время для передачи пакетов длиной до 4 ГБ.
\item Заголовок параметров адресата. В нём содержится информация, которую будет обрабатывать конечный получатель пакета.
\item Заголовок маршрутизации. Используется для маршрутизации. В нём содержится список узлов, через которые должна пройти IP-дейтаграмма. Он начинается: сначала указывается поле следующего заголовка, затем указывается длина заголовка маршрутизации, потом указывается тип маршрутизации, потом оставшиеся сегменты, т. е. оставшиеся узлы, через которые должна пройти дейтаграмма. После этого указывается сам маршрут.
\item Заголовок фрагментации. Используется при необходимости фрагментации дейтаграмм. Фрагментация может быть выполнена только отправителем. Заголовок состоит:

\begin{itemize}
\item следующий заголовок;
\item смещение фрагмента. Длина 13 бит. Смещение измеряется в единицах, кратных 64 битам;
\item 2 бита не используются;
\item флаг ``есть ещё фрагменты'';
\item идентификатор дейтаграммы. Длина 32 бита.
\end{itemize}

\item Заголовок аутентификации.
\item Заголовок шифрования.
\end{enumerate}
\section{Протокол IPsec}
\label{sec-4}


Протокол IPsec обеспечивает защиту обмена данными в сетях за счёт шифрования и (или) аутентификации всего потока данных на уровне IP.

Протокол может работать в двух режимах:
\begin{enumerate}
\item Транспортный. В этом режиме защищаются только данные из IP-дейтаграмм, а заголовок IP-дейтаграммы не защищается.
\item Туннельный. В этом режиме защищается вся IP-дейтаграмма. Для этого к защищённой IP-дейтаграмме добавляется новый IP-заголовок, никак не защищённый. Обычно в нём указывается IP-адрес маршрутизатора или шлюза, который стоит в сети конечного получателя.
\end{enumerate}

IPsec поддерживает два протокола защиты:
\begin{enumerate}
\item Аутентификация AH.
\item Протокол шифрования аутентификации ESP.
\end{enumerate}

Внутри каждого из этих протоколов может использоваться несколько различных алгоритмов.

Дополнительно в протоколе IPsec определён протокол распределения ключей.

Заголовок аутентификации обеспечивает аутентификацию IP-дейтаграмм и проверку целостности данных в нём.

Заголовок состоит из следующих полей:
\begin{enumerate}
\item Следующий заголовок. Длина 8 бит.
\item Длина. Здесь длина заголовка в 32-битных единицах минус 2.
\item Зарезервированных 16 бит.
\item Индекс параметров защиты. Длина 32 бита. Он идентифицирует защищённую связь.
\item Порядковый номер. Длина 32 бита. Порядковый номер дейтаграммы, который был послан по данной защищённой связи.
\item Данные аутентификации. В нём содержится код аутентификации.
\end{enumerate}
\subsection{Защищённая связь}
\label{sec-4_1}


Связь --- односторонние отношения между отправителем и получателем.

Связь определяется параметрами:
\begin{enumerate}
\item Индекс параметров защиты. Строка битов, которая обозначает некий условный номер этой связи. По нему определяются алгоритмы обработки принятого пакета.
\item IP-адрес получателя.
\item Идентификатор протокола защиты. Параметры защищённой связи хранятся в специальных таблицах. В этих таблицах записаны:

\begin{itemize}
\item счётчик порядкового номера;
\item флаг переполнения счётчика порядкового номера;
\item окно защиты от воспроизведения. Для защиты от повторной передачи одних и тех же дейтаграмм.
\end{itemize}

\item Информация AH. Хранятся параметры для алгоритма аутентификации.
\item Информация ESP. В нём хранятся параметры выбранного алгоритма шифрования.
\item Время жизни защищённой связи. Это интервал времени или значение счётчика байтов, по достижении которого связь уничтожается.
\item Режим IPsec.
\item Максимальная единица передачи маршрута. Максимальный размер пакета, который может быть передан без фрагментации.
\end{enumerate}

Защищённые связи связываются с потоком IP через селекторы. Эти селекторы хранятся в базе данных политики защиты. Деление потоков может осуществляться по IP адресам пункта назначения, IP адресам источников, по протоколу транспортного уровня, по метке потока протокола IPv6 и т. п.
\subsection{Формат пакетов ESP}
\label{sec-4_2}


\begin{enumerate}
\item Индекс параметров защиты. Длина 32 бита. Номер защищённой связи.
\item Порядковый номер дейтаграммы. Длина 32 бита.
\item Передаваемые данные.
\item Заполнитель. Нужен для правильной работы алгоритма шифрования.
\item Длина заполнителя.
\item Следующий заголовок. Длина 8 бит.
\item Данные аутентификации. Вычисляется для всей дейтаграммы ESP.
\end{enumerate}

Шифры RC5, тройной DES, IDEA, BlowFish, CAST.
\subsection{Управление ключами}
\label{sec-4_3}


Управление ключами может быть ручное (когда администратор сам вводит ключи в систему) и автоматизированное. Для автоматизированного применяя протокол ISAKMP/OAKLEY. OAKLEY --- протокол управления ключами основан на алгоритме Диффи-Хеллмана.

К обычному Диффи-Хеллману в нём добавлена аутентификация сторон, обменивающихся ключами. Аутентификация может быть выполнена с помощью ЭЦП или алгоритмов шифрования.
\subsection{Протокол ISAKMP}
\label{sec-4_4}


Протокол защищённой связи и управления ключами. Сообщения этого протокола состоят из заголовка и данных. Они передаются с помощью протокола транспортного уровня UDP. В заголовке присутствуют следующие поля:
\begin{enumerate}
\item Случайное число, которое генерируется стороной, изменяющей, создающей, удаляющей связь.
\item Случайное число объекта получателя.
\item Следующий полезный груз. В этом поле указывается тип данных, которые передаются в сообщении ISAKMP.
\item Главный номер версии.
\item Дополнительный номер версии.
\item Тип обмена.
\item Флаги. Флаг указывает зашифрованы или нет данные ISAKMP.
\item Бит фиксации. Он нужен, чтобы удостовериться, что сначала была создана защищённая связь, а потом получены соответствующие пакеты ISAKMP.
\item Универсальный идентификатор сообщения.
\item Длина сообщения в байтах.
\end{enumerate}

Типы полезного груза:
\begin{enumerate}
\item Защищённая связь. Нужна, чтобы начать процесс создания защищённой связи.
\item Тип предложения. В нём указывается применяемый протокол ESP/AH, число трасформаций.
\item Трасформация. В каждой трасформации передаются атрибуты используемого алгоритма шифрования или аутентификации. Трансформаций может быть указано несколько.
\item Тип обмена ключами.
\item Идентификация. Предназначена для аутентификации связывающих сторон.
\item Сертификат. Сертификат открытого ключа.
\item Цифровая подпись.
\item Хеширование.
\item Запрос сертификата.
\item Нонс. Случайное число. Оно нужно, чтобы обеспечить защиту от атак воспроизведения сообщений и обеспечить процесс обмена сообщениями в реальном времени.
\item Тип уведомления.
\item Тип удаления. Удаление защищённой связи.
\end{enumerate}
\subsection{Тип обмена}
\label{sec-4_5}


\begin{enumerate}
\item Базовый обмен. Происходить обмен ключами и данными аутентификации одновременно.
\item Обмен с защитой идентификации сторон.
\item Обмен только данными аутентификации.
\item Обмен без идентификации сторон.
\item Информационный обмен. Нужен для передачи сообщений о параметрах управления защищённой связью.
\end{enumerate}
\section{Протоколы транспортного уровня}
\label{sec-5}
\subsection{TCP}
\label{sec-5_1}


Протокол TCP является пакетным. Пакеты называются сегментами. Каждый сегмент имеет заголовок.

Формат \emph{заголовка}:
\begin{enumerate}
\item Порт отправителя. Длина 16 бит.
\item Порт получателя. Длина 16 бит.
\item Позиция сегмента.
\item Первый ожидаемый байт. Используется только, если сегмент --- это квитанция.
\item Смещение данных. Это длина заголовка в 32-разрядных словах. Длина 4 бита.
\item 6 бит неиспользуемых.
\item Флаги. 6 флагов.

\begin{itemize}
\item URG. Срочность данных.
\item ACH. Квитанция.
\item PSH. Сегмент послать в первую очередь.
\item RST. Запрос на установку первоначальных параметров соединения.
\item SYN. Синхронизация счётчиков переданных данных при установлении соединения.
\item FIN. Отправлен последний бит сообщения.
\end{itemize}

\item Размер окна. В нём указывается сколько байт готов принять получатель.
\item Контрольная сумма. Длина 16 бит. Контрольная сумма вычисляется на весь сегмент + IP адреса отправителя и получателя, идентификатор протокола и длину сегмента.
\item Указатель срочности данных.
\item Опции.

\begin{itemize}
\item 0 --- конец списка опций.
\item 1 --- отсутствие операции.
\item 2 --- максимальный размер сегмента.
\end{itemize}

\item Заполнитель. Дополняет заголовок до целого числа 32-разрядных слов.
\item Поле данных. Размер не фиксирован (максимальный указан в опции максимальный размер сегмента).
\end{enumerate}

\emph{Номер порта} --- число, которое однозначно определяет приложение, осуществляющее сетевой обмен. Каждому приложения записан определённый номер порта.

\emph{Сокет} --- число, в которое входит IP адрес компьютера и номер порта. Однозначно определяет связь между процессами через протокол TCP.

Т. к. TCP отвечает за гарантированную доставку сообщений, поэтому передача сообщения происходит после установления соединения между отправителем и получателем. На каждую переданную дейтаграмму (сегмент) получатель должен послать квитанцию
\subsubsection{Передача сообщения}
\label{sec-5_1_1}


Сообщение от прикладного уровня является потоком, представляет собой последовательность байт фиксированной длины, передаваемых асинхронно.
TCP разбивает этот поток на сегменты и к каждому из них добавляет соответствующий заголовок. Длина сегмента задаётся администратором или определяется автоматически протоколом TCP.

Сначала происходит установление соединения. Отправитель посылает сегмент, в котором содержится номер сокета. В заголовке флаг SYN установлен в единицу. В ответ получатель посылает номер своего сокета. При этом в заголовке установлены флаги SYN и ASK. Отправитель посылает сегмент, в заголовке которого флаг ASK установлен в 1 и в поле, где указывается номер сегмента устанавливается 1. На этом процесс соединения заканчивается.

Если сообщение состоит из нескольких TCP сегментов, то получатель собирает его согласно порядковых номеров, хранящихся в заголовке. Если сегмент потерян или повреждён, то отправителю посылается сообщение, содержащее порядковый номер этого сегмента. Отправитель повторно передаёт запрошенный сегмент. Если сообщение принято, то посылается квитанция.

В последнем сегменте сообщения в заголовке должен быть установлен флаг FIN. После этого соединение разрывается.

Чтобы предотвратить переполнение буфера получателя используется т. н. скользящее окно, т. е. в заголовке передаётся размер окна, который может принять получатель.

В протоколе TCP используется несколько таймеров:
\begin{enumerate}
\item Таймер повтора передачи. Устанавливает время ожидания квитанции. Если квитанция за этот промежуток времени не поступает, сегмент считается потерянным и отправляется вновь. Повторная передача происходит заданное число раз. Если передача не удалась, то на прикладной уровень сообщается об ошибке.
\item Таймер задержки. Нужен, чтобы исключить повторное открытие только что закрытого порта, которое может быть вызвано прибывшими сегментами. Задержка может достигать 30 сек.
\item Таймер запроса. Нужен когда получатель приостановивший приём данных отправляет сообщение о возобновлении работы, но не получает подтверждения. Чтобы продолжить передачу, отправитель посылает запросы с периодом, заданным этим таймером.
\item Таймер контроля. Он вызывает периодическую передачу сегментов без данных. Нужен для проверки сети. Значение между 5--45 секундами.
\item Таймер разъединения. Задаёт максимальное время ожидания ответа. По истечении этого срока соединение разрывается. Максимальное время обычно равно 360 сек.
\end{enumerate}
\subsection{UDP}
\label{sec-5_2}


Это протокол транспортного уровня. Передача данных в нём происходит без установления соединения. Отправителю никак не сообщается доставлено ли его сообщение, правильно ли оно принято. Исправление ошибок происходит либо на сетевом, либо на прикладном уровне. Управление потоком данных не предусмотрено.

Заголовок UPD дейтаграммы:
\begin{enumerate}
\item Порт отправителя. Длина 16 бит. Поле необязательное.
\item Порт получателя. Длина 16 бит. Поле обязательно.
\item Длина дейтаграммы. Длина 16 бит.
\item Контрольная сумма. Длина 16 бит. Вычисляется также, как в протоколе TCP.
\item Данные.
\end{enumerate}
\section{Дополнительная лекция}
\label{sec-6}
\subsection{IP адресация}
\label{sec-6_1}


IP адрес является уникальным 32-битным идентификатором IP интерфейса в сети Интернет, т. е. если у хоста несколько интерфейсов, у него будет несколько IP адресов.

IP адрес принято записывать в десятичном виде с разбивкой 32-битного числа по октетам. IP адрес состоит из 2 частей. Старшие разряды являются адресом сети, младшие разряды --- адресом хоста. Граница разделов 2 частей определяются маской (subnet mask).

\emph{Маска} --- 32-битовая комбинация, в которой единицы установлены на сетевой части адреса, а нули на хостовой.
\subsection{Классовая модель адресов}
\label{sec-6_2}


Существуют 5 классов адресов:
\begin{enumerate}
\item A. 255.0.0.0. Диапазон 0.0.0.0 - 127.0.0.0
\item B. 255.255.0.0. Диапазон 128.0.0.0 - 192.255.0.0
\item C. 255.255.255.0. Диапазон 193.0.0.0 - 223.255.255.255.0
\item D. Сеть мультиадресной рассылки. Адреса этого диапазона могут быть присвоены нескольким сетевым интерфейсам. Диапазон 224.0.0.0 - 239.0.0.0
\item E. Диапазон 240.0.0.0 - 255.255.255.255
\end{enumerate}
\subsection{Специальные IP адреса}
\label{sec-6_3}


\begin{itemize}
\item 0.0.0.0 --- маршрут по умолчанию (default road). Используется в маршрутных таблицах для указания направления передачи пакетов, адресат которых неизвестен.
\item 255.255.255.255 --- широковещательный адрес (broadcast) локальной сети, в которой абонент находится.
\item адрес, у которого хостовая часть нулевая называется адресом сети и он не может быть присвоен никакому хосту.
\item адрес, у которого хостовая часть единицы называется широковещательным адресом удалённой сети. Он не может быть присвоен хосту.
\item 127.0.0.0 --- сеть обратной связи (loopback). В ней определён только один интерфейс --- 127.0.0.1. Любой пакет, отправленный по адресу 127.0.0.1 будет принят этим же узлом так, как если бы он пришёл из сети. Используется для отладки сетевых сервисов без подключения к реальной сети.
\end{itemize}
\subsection{Серые адреса}
\label{sec-6_4}


Любой пакет, отправленный по серому адресу будет отброшен маршрутизаторами сети Интернет и останется в пределах локальной сети. Поэтому адреса из серых диапазонов могут иметь несколько хостов в разных локальных сетях.

\begin{itemize}
\item A: 10.0.0.0
\item B: 172.16.0.0 - 172.32.0.0
\item C: 192.168.0.0 - 172.168.255.0
\end{itemize}

Для доступа с серого адреса к сети Интернет используется специальное устройство --- прокси сервер, которое реализует функции трансляции адресов NAT.
\subsection{Бесклассовая модель}
\label{sec-6_5}


1000000 128

1100000 192

1110000 224

1111000 240

1111100 248

1111110 254

Для получения адреса сети необходимо IP адрес узла в двоичном виде поразрядно умножить на маску. Для получения адреса хоста IP адрес в двоичном виде поразрядно необходимо умножить на инвертированную маску

Сеть 172.16.40.0/24

3 сети 20 хостов

172.16.40.000/00000

172.16.40.001/00000

172.16.40.010/00000

Диапазон
0: 172.16.40.000/00001 = 1

   172.16.40.000/11110 = 30

1: 172.16.40.001/00001 = 33

   172.16.40.001/11110 = 62

2: 172.16.40.010/00001 = 65

   172.16.40.010/11110 = 94
\subsection{Практика}
\label{sec-6_6}


65.179.19.241 

255.255.128.0 маска

Найти адрес сети, хоста, диапазон, широковещательный адрес сети.

Умножить 19 на 128 в двоичном виде поразрядно. 

Адрес сети: 65.179.0.0

128 инвертировать и умножить.

Адрес хоста: 0.0.19.241

Адрес сети: минимальный 65.179.00000000.0000001

максимальный 65.179.127.255

Широковещательный: 65.179.0.255
\subsection{Маршрутизация}
\label{sec-6_7}



\begin{center}
\begin{tabular}{rrrrr}
   Адрес пол.  &       Маска пол.  &  Маршрутизатор  &     Интерфейс  &  Метрика  \\
\hline
  172.16.40.0  &  255.255.255.224  &    172.16.40.1  &   172.16.40.1  &        1  \\
 172.16.40.32  &  255.255.255.224  &   172.16.40.33  &  172.16.40.33  &        1  \\
 172.16.40.64  &  255.255.255.224  &   172.16.40.65  &  172.16.40.65  &        1  \\
      0.0.0.0  &          0.0.0.0  &    65.137.80.1  &  65.137.80.11  &        1  \\
\end{tabular}
\end{center}
\section{Протокол SSL}
\label{sec-7}


Протокол SSL предназначен для защищённой передачи данных через протокол TCP. Протокол SSL в стеке протоколов находится над TCP, но ниже протоколов прикладного уровня.

Протокол прикладного уровня, который используют SSL --- HTTP.

SSL состоит из 4 отдельных протоколов:
\emph{Протокол записи}. Этот протокол непосредственно взаимодействует с протоколом TCP. Он обеспечивает конфиденциальность сообщений и целостность сообщений. Конфиденциальность обеспечивается за счёт шифрования данных, а целостность сообщений за счёт добавления к данным кода аутентичности (контрольная сумма). Данные приложения разбиваются на фрагменты размером не более 2$^{\mathrm{14}}$ байт. Потом может быть выполнено сжатие данных. На эти данные вычисляется код аутентичности и этот код добавляется после данных.

Сформированный блок шифруется с использованием симметричного алгоритма. К полученному зашифрованному блоку добавляется заголовок протокола записи.
Сформированный пакет поступает на уровень протокола TCP.

Алгоритмы шифрования DES. Код аутентичности SHA-1.

Поля заголовка:
\begin{enumerate}
\item Тип содержимого. Длина 8 бит. Определяется протокол лежащего выше уровня, которому адресован фрагмент.
\item Главный номер версии. Длина 8 бит. Главный номер версии используемого протокола SSL.
\item Дополнительное поле. Длина 8 бит.
\item Дополнительный номер протокола.
\item Длина сжатого фрагмента. Длина 16 бит. Длина в байтах фрагмента открытого текста. Максимальное значение 2$^{\mathrm{14}}$ + 2048.
\end{enumerate}

\emph{Протокол изменения параметров шифрования}. Этот протокол расположен над протоколом записи. Этот протокол служит для передачи сообщения с параметром скопировать состояние ожидания в текущее состояние в результате чего обновляются параметры шифров, используемых для данного соединения. Сообщение представляет собой 00000001.

\emph{Протокол извещения}. Он предназначен для обмена служебными сообщениями о работе SSL. Он также расположен над протоколом записи. Сообщение состоит из 2 байт. Первый байт означает уровень предупреждения или уровень неустранимой ошибки. Если уровень равен 2, то соединение по протоколу SSL разрывается. Второй байт --- код, означающий смысл извещения.

\emph{Протокол квитирования}. Лежит над протоколом записи. По этому протоколу происходит взаимная аутентификация сторон, согласовываются алгоритмы шифрования и их параметры, алгоритмы вычисления кода аутентичности, передаются ключи. Этот протокол используется до начала передачи данных. Сообщение протокола состоит из 3 частей:
\begin{enumerate}
\item Тип сообщения. Размер 1 байт.
\item Длина сообщения. Размер 3 байта.
\item Содержимое.
\end{enumerate}
\subsection{Работа протокола SSL}
\label{sec-7_1}

\emph{Определение характеристик защиты}. Процесс передачи инициируется клиентом. Для этого он отправляет серверу сообщение ``clien hello''. В качестве параметров этого сообщения указываются наивысший номер версии протокола, поддерживаемый клиентом; случайное число --- используется во время обмена ключами для защиты от атак воспроизведения; идентификатор сеанса --- сеанс в протоколе ssl это связь между клиентом и сервером; список шифров, которые поддерживает клиент; список методов сжатия, которые поддерживает клиент. В ответ на это сообщение сервер должен послать сообщение ``server hello'', в котором будут те же параметры. В качестве случайного числа передаётся число, сгенерированное сервером.

Возможные методы обмена ключами --- RSA, Диффи-Хелмана и его различные модификации.

Алгоритмы шифрование RC4, RC2, DES, IDEA, 3DES, DES40.

Вычисление кода проверки целостности MD5, SHA-1.

\emph{Аутентификация и обмен ключами}. Если требуется аутентификация сервера, сервер отправляет свой сертификат X.509. Затем передаётся необходимая информация для выработки общего сеансового ключа с клиентом. В конце этой информации сервер отправляем ``server done''. Получив это сообщение клиент проверяет подлинность сертификата. После клиент отправляет свой сертификат серверу и необходимую со своей стороны информацию для выработки общего сеансового ключа.

\emph{Завершение создания защищённого соединения}. Клиент отправляет сообщение изменение параметров шифрования, т. е. начинает работать протокол изменения параметров шифрования.

Затем сразу же отправляется сообщение ``finished'', которое зашифровано на выбранном алгоритмом с сформированным сеансовым ключом. В ответ на эти два сообщения сервер посылает своё сообщение изменение параметров шифрования и своё сообщение ``finished'', зашифрованное при помощи ключа. Эти сообщения нужны для того, чтобы узнать, что у клиента и сервера один сеансовый ключ.

Если установление сеанса завершилось успешно, начинается передачи данных от протоколов вышележащих уровней.
\subsection{Уязвимости SSL}
\label{sec-7_2}


\begin{enumerate}
\item Вскрытие используемых алгоритмов шифрования.
\item Уязвимость к атакам открытого текста. Эта атака используется для определения сеансового ключа. Она может успешна из-за того в открытом тексте часто встречаются одни и те же команды протокола HTTP.
\item Атака воспроизведения. Противник пытается передать серверу заранее перехваченное сообщение от клиента. Успешность данной атаки определяется длиной случайного числа, являющегося идентификатором сеанса.
\item Атака ``посредник''. Противник между клиентом и сервером.
\end{enumerate}
\section{Протокол Kerberos}
\label{sec-8}


Этот протокол предназначен для аутентификации и обмена ключами, которые нужны для установки защищённого канала связи между абонентами, работающими в Интернете.

Этот протокол является протоколом прикладного уровня и разработан для сетей TCP/IP.

Kerberos состоит:
\begin{enumerate}
\item Сервер аутентификации.
\item Сервер выдачи мандатов.
\item Клиенты.
\item Серверы, к которым пользователи (клиенты) обращаются за каким либо ресурсом.
\end{enumerate}

Подстроен на основе протокола Нитхема/Шредера с 3-ей доверенной стороной. Первый 2 компонента являются этой стороной. 

На сервере аутентификации хранится БД, в которой записаны идентификаторы всех пользователей сети и их секретные ключи, идентификаторы всех ресурсов сети и их секретные ключи. Эти секретные ключи позволяют шифровать сообщения для клиентов и серверов. Успешное расшифрование этих сообщений является гарантией прохождения аутентификации всеми участниками протокола. 
\subsection{Описание протокола}
\label{sec-8_1}


\begin{enumerate}
\item Клиент посылает серверу аутентификации сообщения с запросом на разрешение доступа к серверу выдачи мандатов. Это сообщение включает в себя:

\begin{itemize}
\item идентификатор клиента,
\item идентификатор сервера выдачи мандатов и
\item метку времени.
\end{itemize}

\item Сервер аутентификации отвечает клиенту в сообщении, которое зашифровано секретным ключом клиента, который хранился в БД. В этом сообщении содержится:

\begin{itemize}
\item сеансовый ключ для связи с сервером выдачи мандатов,
\item идентификатор сервера выдачи мандатов,
\item метку времени, когда был отправлен ответ, срок действия мандата и мандат сервера выдачи мандата. \emph{Мандат} --- специальная информация, на основе которой происходит проверка подлинности обращающего субъекта.
\end{itemize}

\item Клиент посылает полученный мандат и идентификатор требуемого ему сервиса серверу выдачи мандатов. В этом сообщении присутствует \emph{аутентификатор клиента}. Аутентификатор клиента из идентификатора клиента, сетевого адреса и метки времени. Эти 3 компонента зашифровываются на ключе, который был получен на шаге 2.
\item Сервер выдачи мандатов расшифровывает полученный аутентификатор клиента, проверяет разрешён ли клиенту доступ к запрашиваемому ресурсу и если разрешён, посылает сообщение, зашифрованное тем же ключом, полученным в шаге 2. Сообщение состоит из:

\begin{itemize}
\item ключа для установления связи между клиентом и запрашиваемым сервисом,
\item идентификатора сервиса,
\item метки времени и мандата сервиса. \emph{Мандат сервиса} --- зашифрованное сообщение при помощи ключа связи между сервисом и сервером выдачи мандатов. Внутри этого сообщения находится:

\begin{itemize}
\item сеансовый ключ для связи клиента и сервиса,
\item идентификатор клиента,
\item сетевой адрес клиента,
\item идентификатор сервиса,
\item метка времени,
\item срок действия мандата.
\end{itemize}

\end{itemize}

\item Клиент передаёт сервису полученный на шаге 4 мандат и свой аутентификатор. Аутентификатор на этом шаге шифруется при помощи ключа, полученного на шаге 4.
\item Сервис проверяет полученное сообщение. Если процедуры расшифрования прошли успешно, то аутентификация прошла успешно, т. е. сервис удостоверился, что к нему обращается клиент, указанный в сообщении. Происходит, если требуется аутентификация сервиса. Сервис отсылает клиенту зашифрованное сообщение, полученную на шаге 5 метку времени + 1-ца. Используется ключ тот же, что и на шаге 5 для шифрования аутентификатора.
\end{enumerate}

Среда Kerberos должна удовлетворять следующим условиям:
\begin{enumerate}
\item Сервер аутентификации должен хранить свои байты данных, хешированные пароли всех пользователей системы. Пароли пользователей используются для формирования идентификатора клиента.
\item Все сервисы в сети должны быть зарегистрированы и у каждого из них должен быть свой секретный ключ для связи с сервером Kerberos.
\end{enumerate}

Kerberos работает в пределах одной локальной сети. Если пользователю требуются ресурсы другой сети, то Kerberos доступа к ним не разрешит. Чтобы это устранить необходимо, чтобы оба сервера Kerberos были зарегистрированы друг в друге. Соответственно для связи между серверами должны быть установлены секретные ключи. При такой конфигурации клиент сначала в своей сети должен получить мандат доступа к серверу выдачи мандатов в другой сети. После этого клиент обращается к серверу выдачи мандатов другой сети и получает доступ к интересующему ресурсу.

Отличия 4 и 5 версии Kerberos: в версии 4 использовался алгоритм шифрования DES. В версии 5 может быть выбран любой другой алгоритм шифрования. В версии 4 требуется использование IP-адрасации, в версии 5 --- любые сетевые адреса. В версии 4 срок действия мандата составлял 1280 минут максимум, потому, что срок действия мандата представлялся 8-битовым числом; в версии 5 явно указывается момент начала действия мандата и момент его окончания.

Уязвимости Kerberos:
\begin{enumerate}
\item Повторное использование перехваченной информации.
\item Синхронизация часов. Т. е. система будет работать правильно, если у всех её участников часы синхронизированы.
\item Сложность паролей.
\item Повторное использование идентификаторов субъектов. Новый объект системы может получить идентификатор выбывшего.
\item Сеансовые ключи. Один и тот же ключ используется в нескольких сеансах связи.
\end{enumerate}

Kerberos в настоящее время поддерживается Windows и FreeBSD.
\section{Межсетевые экраны}
\label{sec-9}


Межсетевые экраны реализуют методы контроля за информацией, поступающей или выходящей из системы.

Защита системы обеспечивается за счёт фильтрации информации на основе критериев, заданных администратором.

Процедура фильтрации включает в себя анализ заголовков каждого пакета, проходящего через экран и передача их дальше только в том случае, если они удовлетворяют правилам фильтрации. Если не удовлетворяет, то пакет уничтожается. 

Фильтрация пакетов может выполняться для протоколов разных иерархических уровней (сетевого, транспортного, прикладного).

\begin{enumerate}
\item На сетевом уровне в качестве критериев фильтрации используются IP-адреса отправителя и получателя, тип данных в пакете ICMP протокола и т. п.
\item На транспортном уровне --- номера портов отправителя, получателя, флаги в TCP-сегментах, поле \emph{длина} в TCP сегменте.
\item На прикладном уровне --- типы команд протокола, длины заголовков, адреса ресурсов.
\end{enumerate}

Правила фильтрации могут быть настроены 2 способами:
\begin{enumerate}
\item Всё, что не запрещено --- разрешено.
\item Всё, что не разрешено --- запрещено.
\end{enumerate}

Межсетевые экраны позволяют скрывать реальные IP адреса в защищаемой системе при помощи функции трансляции сетевых адресов (NAT).

При поступлении пакета данных межсетевой экран заменяет реальный IP адрес отправителя пакета на виртуальный и посылает его получателю. 

\begin{enumerate}
\item Виртуальный IP адрес может быть неизменным, т. е. одинаковый для всех адресатов. В этом случае межсетевой экран каждому соединению присваивает уникальное числовое значение (номера портов).
\item Для каждого узла защищаемой сети может быть выделен отдельный виртуальный адрес. При этом эти адреса периодически изменяются.
\end{enumerate}

Межсетевые экраны могут реализованы аппаратно или программно. Аппаратные экраны обычно устанавливаются в точке подключения защищаемой сети к Интернету. Программные реализации экранов устанавливаются на серверы, рабочие станции, некоторые типы маршрутизаторов и коммутаторов.

\end{document}
